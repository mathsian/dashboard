\errorcontextlines 10000
\documentclass[a4paper,12pt,portrait]{article}
\usepackage{graphicx}
% for text wrapping in table
\usepackage{tabularx}
% for footer
\usepackage[]{fancyhdr}
% never hyphenate in this document
\usepackage[none]{hyphenat}
% set up fonts
\usepackage{fontspec}
\setsansfont{Titillium-Regular}
\setmainfont{Titillium-Regular}
\setromanfont{FSSinclair-Medium}

% smaller margins
\usepackage[margin=1.5cm]{geometry}

% custom colours
\usepackage{xcolor}
\definecolor{adaYellow}{RGB}{245,225,52}
\definecolor{adaPurple}{RGB}{149,96,159}
\definecolor{adaGreen}{RGB}{161,200,84}
\definecolor{adaCoral}{RGB}{236,98,113}
\definecolor{adaBlue}{RGB}{108,184,231}
\definecolor{adaOrange}{RGB}{246,131,82}

\definecolor{adaBlack}{RGB}{9,20,8}
\definecolor{adaGrey}{RGB}{134,136,140}
\definecolor{adaLightGrey}{RGB}{211,212,211}
\definecolor{adaWhite}{RGB}{255,255,255}

% customise sections
% need explicit to access parameter number to get the section on the left
\usepackage[explicit]{titlesec}
\titleformat{\section}[hang]{\rmfamily\uppercase}{\thesection.}{0em}{ #1\hrulefill}
\titleformat{\subsection}[hang]{\rmfamily\uppercase}{\thesubsection.}{0em}{ #1}

\begin{document}

\cfoot{\sffamily \small \color{adaGrey} ada.ac.uk  |  info@ada.ac.uk  |  +44 203 105 0125  | charity no. 1158399}
\renewcommand{\headrulewidth}{0pt} % remove header rule
\pagestyle{fancy}

\begin{flushright}
\includegraphics[height=3cm]{NCDS.jpg}\\
  {\uppercase{\small Ada. National College for Digital Skills\\
      Broad Lane, Tottenham Hale\\
      London N15 4AG}}
\end{flushright}

%\section*{ACADEMIC TRANSCRIPT}

\begin{tabular}{l l}
\uppercase{Student name} & \textsf{\VAR{student_name}}\\
\uppercase{Student ID} & \textsf{\VAR{student_id}}\\
\uppercase{Programme} & \textsf{\VAR{programme}}\\
\uppercase{Issued} & \textsf{\VAR{issued}}
\end{tabular}

\begin{center}
\begin{tabular}{l l r r l}
\uppercase{Module name} & \uppercase{Module code} & \uppercase{Started} & \uppercase{Total} & \uppercase{Class}\\
\hline
%% for level in modules
\uppercase{Level \VAR{level.level}} \\
%% for module in level.results
\textsf{\VAR{module.name}} & \textsf{\VAR{module.code}} & \textsf{\VAR{module.first}} & \textsf{\VAR{module.total}} & \textsf{\VAR{module.class}}\\
%% endfor
%% endfor
\hline
&&& \textbf{\textsf{\VAR{overall}}} & \textbf{\textsf{\VAR{overall_class}}}
\end{tabular}
\end{center}

\end{document}