\errorcontextlines 10000
\documentclass[a4paper,12pt]{article}
\usepackage{etoolbox}
% for flexible tables
\usepackage{tabularx}
\newcolumntype{b}{X}
\newcolumntype{m}{>{\hsize=.4\hsize}X}
\newcolumntype{s}{>{\hsize=.2\hsize}X}

% set up fonts
\usepackage{fontspec}
\setsansfont{Titillium-Regular}
\setmainfont{Titillium-Regular}
\setromanfont{FSSinclair-Medium}

% smaller margins
\usepackage[margin=1cm]{geometry}

% customise sections
% need explicit to access parameter number to get the section on the left
\usepackage[explicit]{titlesec}
\titleformat{\section}[hang]{\rmfamily\uppercase}{\thesection.}{0em}{ #1\hrulefill}
\titleformat{\subsection}[hang]{\rmfamily\uppercase}{\thesubsection.}{0em}{ #1}
\titleformat{\subsubsection}[hang]{\sffamily}{\thesubsubsection.}{0em}{ #1}
% drawing packages
\usepackage{tikz}
\usetikzlibrary{shapes}
\usetikzlibrary{decorations.text}
\usepackage{pgfplots}
\pgfplotsset{
  compat = newest,
  tick label style = {font=\sffamily},
  every axis label = {font=\sffamily},
  legend style = {font=\sffamily\small, draw=none},
  y axis line style = {draw=none},
  label style = {font=\sffamily}
}

% custom colours
\definecolor{adaYellow}{RGB}{245,225,52}
\definecolor{adaPurple}{RGB}{149,96,159}
\definecolor{adaGreen}{RGB}{161,200,84}
\definecolor{adaCoral}{RGB}{236,98,113}
\definecolor{adaBlue}{RGB}{108,184,231}
\definecolor{adaOrange}{RGB}{246,131,82}

\definecolor{adaBlack}{RGB}{9,20,8}
\definecolor{adaGrey}{RGB}{134,136,140}
\definecolor{adaLightGrey}{RGB}{211,212,211}
\definecolor{adaWhite}{RGB}{255,255,255}

\definecolor{clr1}{RGB}{149,96,159}
\definecolor{clr2}{RGB}{161,200,84}
\definecolor{clr3}{RGB}{236,98,113}
\definecolor{clr4}{RGB}{108,184,231}
\definecolor{clr5}{RGB}{246,131,82}

% Declare layers
\pgfdeclarelayer{background}
\pgfsetlayers{background,main}

\setlength{\columnsep}{1em}
\renewcommand{\arraystretch}{1.2}

\begin{document}
\sffamily
% no page numbers
\pagestyle{empty}
% title
\begin{center}
    {\rmfamily\uppercase{
        {\large \VAR{name}}\\
        \VAR{date}\\
        Team \VAR{team}}}
\tikz[remember picture,overlay,shift=(current page.north east)] \node[inner sep=0pt] at (-2,-2) {\includegraphics[height=2cm]{NCDS.jpg}};
\end{center}
% Attendance bar chart
\section*{Attendance}
\noindent \begin{tabular}{r l r l}
Attendance & \VAR{attendance}\% & Punctuality & \VAR{punctuality}\%
\end{tabular}


% \begin{tikzpicture}[scale=1]
%   \begin{axis}[
%   width=10cm,
%   height=5cm,
%   ybar stacked,
%   legend style={{at=(1.2,0.5)},anchor=west},
%   legend cell align={left},
%   axis y line*=left,
%   axis x line*=bottom,
%   symbolic x coords={0,\VAR{attendance_dates},1},
%   xtick ={\VAR{attendance_dates}},
%   tick align=outside,
%   x tick label style={rotate=90,anchor=east},
%   y tick label style={anchor=east},
%   ymin = 0,
%   ymax = 100,
%   ytick = {95, 100},
%   yticklabels = {},
%   xticklabels = {\VAR{attendance_dates}},
%   xmin = 0,
%   xmax = 1,
%   ylabel = \% sessions attended,
%   grid=major,
%   clip=false,
%   ]
%   \addplot[adaBlue, bar width=14pt, fill=adaBlue] coordinates {\VAR{attendance_zip}};
%   \addplot[adaPurple, bar width=14pt, fill=adaPurple] coordinates {\VAR{punctuality_zip}};
%   \node at (axis cs:1,95) [anchor=west] {{\sffamily Target 95\%}};
%   \legend{Present,Present but late}
% \end{axis}
% \end{tikzpicture}
% Academic
\section*{Academic}
%% for subject, assessment in academic.items()
\subsection*{\rmfamily \VAR{subject}}

% old table style
%% if 0 > 1
\noindent\begin{tabularx}{\linewidth}{l l X}
%% for assessment_name, assessment_details in assessment.items()
            \VAR{assessment_name} & \VAR{assessment_details['grade']} \\
            %% if assessment_details['subtitle']
            \multicolumn{3}{p{\hsize}}{\VAR{assessment_details['subtitle']}}\\
            %% endif
            %% if assessment_details['date'] > '2021-08-30'
            \multicolumn{3}{p{\hsize}}{\em \VAR{assessment_details['comment']}}\\
            %% endif
%% endfor
\end{tabularx}
%% endif

% non-table style
%% if 2 > 1
%% for assessment_name, assessment_details in assessment.items()

\subsubsection*{\VAR{assessment_name}: \VAR{assessment_details['grade']}}

%% if assessment_details['comment']|length > 0 and assessment_details['report'] == 0
\begin{quotation}
{\em \VAR{assessment_details['comment']}}
\end{quotation}
%% endif

%% endfor
%% endif

%% endfor

% Kudos start
%% if 2 > 1

% Kudos
\section*{\VAR{kudos_total} Kudos}
\noindent Kudos points are awarded to students who demonstrate the college
values of curiosity, creativity, collaboration, rigour and resilience

\vspace{1em}
%% if kudos|length > 0
\noindent\begin{tabularx}{\linewidth}{l l X}
%% for kudo in kudos
\VAR{kudo.date}& $\VAR{kudo.points}\times$ \VAR{kudo.ada_value} & {\em
                                                                  \VAR{kudo.description}}
                                                                  \mbox{-- \VAR{kudo.from}}\\
%% endfor
    \end{tabularx}
%% else
\noindent No kudos points have been recorded this term
%% endif

%% endif
% Kudos end

% Concerns start
%% if 0 > 1

% Concern
\section*{\VAR{concern_total} Concerns}
\noindent Concerns may be raised by staff about a student's conduct, their
academic performance or their attendance

\vspace{1em}
%% if concerns|length > 0
\noindent\begin{tabularx}{\linewidth}{l l X}
%% for concern in concerns
 \VAR{concern.date}& \VAR{concern.category} & {\em \VAR{concern.description}} -- \VAR{concern.from}\\
  %% endfor
     \end{tabularx}
  %% else
\noindent No concerns have been recorded
  %% endif


% \section*{Coming up}
% \begin{tabularx}{\linewidth}{l l l X}
%   9th July & & & Last day of Summer term \\
%            & & & End of year celebration \\
%   27th Aug & & & Re-enrolment day \\
%   31st Aug & -- & 17th Dec & Autumn term \\
%   18th Oct & -- & 29th Oct & Half term break
% \end{tabularx}
% Messages
%% if messages
\section*{Messages}
\noindent\begin{tabularx}{\linewidth}{l X}
%% for message in messages
           \VAR{message.title} & \VAR{message.message}\\
           %% endfor
         \end{tabularx}
%% endif
% Concerns end
%% endif

\end{document}
